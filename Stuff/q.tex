%% LyX 2.0.6 created this file. For more info, see http://www.lyx.org/.
%% Do not edit unless you really know what you are doing.
\documentclass{article}
\usepackage[T1]{fontenc}
\usepackage[landscape]{geometry}
\geometry{verbose,tmargin=0cm,bmargin=0cm,lmargin=1cm,rmargin=1cm}
\setlength{\parskip}{0in}
\setlength{\parindent}{0pt}
\usepackage{url}
\usepackage[unicode=true,pdfusetitle,
 bookmarks=true,bookmarksnumbered=false,bookmarksopen=false,
 breaklinks=false,pdfborder={0 0 1},backref=false,colorlinks=false]
 {hyperref}

\makeatletter

%%%%%%%%%%%%%%%%%%%%%%%%%%%%%% LyX specific LaTeX commands.
%% Because html converters don't know tabularnewline
\providecommand{\tabularnewline}{\\}

%%%%%%%%%%%%%%%%%%%%%%%%%%%%%% User specified LaTeX commands.
\usepackage{multicol}
\IfFileExists{upquote.sty}{\usepackage{upquote}}{}
% \VignetteIndexEntry{knitr Reference Card}
% \VignetteEngine{knitr::knitr}

\makeatother

\usepackage{Sweave}
\begin{document}
\Sconcordance{concordance:q.tex:q.Rnw:%
1 28 1 1 0 54 1 1 4 80 1}



\title{knitr Reference Card}


\author{Yihui Xie}

\maketitle
\begin{multicols}{2}


\section{Syntax}

\begin{small}

\noindent \begin{center}
\begin{tabular}{l|l|l|l|l}
\hline
format & start & end & inline & output\tabularnewline
\hline
Rnw & \verb|<<*>>=| & \texttt{@} & \verb|\|\verb|Sexpr{x}| & \TeX{}\tabularnewline
\hline
Rmd & \verb|```{r *}| & \verb|```| & \verb|`r x`| & MD\tabularnewline
\hline
Rhtml & \verb|<!--begin.rcode *| & \verb|end.rcode-->| & \verb|<!--rinline x-->| & HTML\tabularnewline
\hline
Rrst & \texttt{..~\{r {*}\}} & \texttt{..~..} & \verb|:r:`x`| & reST\tabularnewline
\hline
Rtex & \texttt{\% begin.rcode {*}} & \texttt{\% end.rcode} & \verb|\rinline{x}| & \TeX{}\tabularnewline
\hline
Rasciidoc & \texttt{// begin.rcode {*}} & \texttt{// end.rcode} & \texttt{+r x+} & AsciiDoc\tabularnewline
\hline
Rtextile & \texttt{\#\#\# begin.rcode {*}} & \texttt{\#\#\# end.rcode} & \texttt{@r x@} & Textile\tabularnewline
\hline
brew & & & \verb|<% x %>| & text\tabularnewline
\hline
\end{tabular}
\par\end{center}

\end{small}

\texttt{{*}} denotes local chunk options, e.g. \verb|<<label, eval=FALSE>>=|;
\texttt{x} denotes inline R code, e.g. \verb|`r 1+2`| (MD stands
for \href{http://en.wikipedia.org/wiki/Markdown}{Markdown})


\section{Minimal Examples}

\begin{multicols}{2}


\subsection{Sweave ({*}.Rnw)}

\begin{verbatim}
\documentclass{article}
\begin{document}

Below is a code chunk.
<<foo, echo=TRUE>>=
z = 1+1
plot(cars)
@

The value of z is Sexpr{z}.
\end{document}
\end{verbatim}


\subsection{R Markdown ({*}.Rmd)}

\begin{verbatim}
Hi _markdown_!

```{r foo, echo=TRUE}
z = 1+1
plot(cars)
```

The value of z is `r z`.
\end{verbatim}


\subsection{Brew ({*}.brew)}

\begin{verbatim}
The value of pi is <% pi %>.
\end{verbatim}

\end{multicols}


\section{Chunk Options}

\texttt{opts\_chunk} controls global chunk options, e.g. \texttt{opts\_chunk\$set(tidy
= FALSE)}, which can be overridden by local chunk options. See all
options at \url{http://yihui.name/knitr/options}; some frequently
used options:
\begin{description}
\item [{eval}] whether to evaluate the chunk
\item [{echo}] whether to echo source code
\item [{results}] \texttt{'markup'}, \texttt{'asis'}, \texttt{'hold'},
\texttt{'hide'}
\item [{tidy}] whether to reformat R code
\item [{cache}] whether to cache results
\item [{fig.width,\ fig.height,\ out.width,\ out.height}] device and
output size of figures
\item [{include}] whether to include the chunk results in output
\item [{child}] filenames of child documents
\item [{engine}] language name (R, python, ...)
\end{description}

\section{Functions}
\begin{description}
\item [{knit()}] the main function in this package; knit input document
and write output
\item [{purl()}] extract R code from an input document
\item [{spin()}] spin goat's hair (an R script with roxygen comments) into
wool (a literate programming document to be passed to \texttt{knit()})
\item [{stitch()}] insert an R script into a template and compile the document
\item [{knit\_hooks\$set()}] set or reset chunk and output \href{http://yihui.name/knitr/hooks}{hooks}
\end{description}

\section{Resources}
\begin{itemize}
\item homepage: \url{http://yihui.name/knitr}
\item development repository: \url{https://github.com/yihui/knitr} (\href{http://cran.r-project.org/package=knitr}{CRAN},
\href{http://rforge.net/knitr}{Rforge})
\item examples: \url{https://github.com/yihui/knitr-examples}
\item stackoverflow: \url{http://stackoverflow.com/questions/tagged/knitr}
\item mailing list: \url{https://groups.google.com/group/knitr}
\end{itemize}
\end{multicols}

\end{document}
